\documentclass[letterpaper,11pt]{article}
\usepackage[spanish]{babel}
\spanishdecimal{.}
\usepackage[utf8]{inputenc}
\usepackage{graphicx}
\usepackage[top=2.5cm, bottom=3cm, left=2.5cm, right=3cm]{geometry}
\usepackage{hyperref}
\usepackage{verbatim}
\usepackage[natbibapa]{apacite}
\usepackage{multicol}
\usepackage{amsmath}
\usepackage{listings}
\usepackage{xcolor}
\bibliographystyle{apacite}

\definecolor{graywhite}{rgb}{0.9529,0.9607,0.9686}
\definecolor{bluegray}{rgb}{0.6823, 0.7411, 0.8}
\definecolor{darkred}{rgb}{0.7372, 0.2392, 0.2392}
\definecolor{bluedark}{rgb}{0.294, 0.4705, 0.6407}
\definecolor{darkgreen}{rgb}{0.1764, 0.5294, 0.0901}
\lstset{
  backgroundcolor=\color{graywhite},   % choose the background color; you must add \usepackage{color} or \usepackage{xcolor}; should come as last argument
  basicstyle=\footnotesize,        % the size of the fonts that are used for the code
  breakatwhitespace=false,         % sets if automatic breaks should only happen at whitespace
  breaklines=true,                 % sets automatic line breaking
  captionpos=b,                    % sets the caption-position to bottom
  commentstyle=\color{darkgreen},    % comment style
  keepspaces=true,                 % keeps spaces in text, useful for keeping indentation of code (possibly needs columns=flexible)
  keywordstyle=\color{darkred},       % keyword style
  language=Octave,                 % the language of the code
  morekeywords={*,...},            % if you want to add more keywords to the set
  numbers=left,                    % where to put the line-numbers; possible values are (none, left, right)
  numbersep=7pt,                   % how far the line-numbers are from the code
  numberstyle=\tiny\color{bluedark}, % the style that is used for the line-numbers
  showspaces=false,                % show spaces everywhere adding particular underscores; it overrides 'showstringspaces'
  showstringspaces=false,          % underline spaces within strings only
  showtabs=false,                  % show tabs within strings adding particular underscores
  stepnumber=1,                    % the step between two line-numbers. If it's 1, each line will be numbered
  stringstyle=\color{bluedark},     % string literal style
  frame=single,
  rulecolor=\color{bluegray},
  tabsize=2,                   % sets default tabsize to 2 spaces
  xleftmargin=1cm,
  xrightmargin=0.5cm,
  framexleftmargin=0.5cm,
  extendedchars=true,
  literate={á}{{\'a}}1 {é}{{\'e}}1 {í}{{\'i}}1 {ó}{{\'o}}1 {ú}{{\'u}}1 {Á}{{\'A}}1 {É}{{\'E}}1 {Í}{{\'I}}1 {Ó}{{\'O}}1 {Ú}{{\'U}}1,
}


\title{Práctica 04\\Detector de esquinas de Harris}
\author{Visión Computacional Aplicada a la Robótica}
\date{UNAM, 2023-2}
\begin{document}
\renewcommand{\tablename}{Tabla}

\maketitle

\subsection*{Resumen}
El alumno implementará el detector de esquinas de Harris sobre una imagen binaria resultante de una detección de bordes.


\subsection*{Objetivos}
\begin{itemize}
\item Aplicar el detector de bordes de Canny de la práctica 02.
\item Aplicar el concepto de gradiente y matriz de segundo momento para detectar esquinas.
\item Aplicar el detector de esquinas de Harris sobre un fluo de video.  
\end{itemize}

\section*{Duración}
1 semanas

\section*{Desarrollo}
Programe en Python o C++ el detector de esquinas de Harris, que se resume en los siguientes pasos:

\begin{enumerate}
\item Conversión de la imagen a escala de grises
\item Obtención de las derivadas parciales $I_x$ e $I_y$
\item Obtención de la matriz de segundo momento empleando una ventana de tamaño $W$.
\item Cálculo de la respuesta de Harris
\item Supresión de no máximos
\end{enumerate}

El programa de detección de esquinas debe dibujar sobre la imagen original marcadores que indiquen claramente la esquina detectada. 

\textbf{Importante:} No se permite el uso de las funciones de OpenCV que ya realizan lo que se pide: cv2.cornerHarris y cv2.goodFeaturesToTrack. Se puede utilizar la función cv2.Sobel para la obtención de las derivadas parciales.
\[\]
Se puede utilizar como base el programa de ejemplo \texttt{harris\_example.cpp}.

\section*{Entregables:}
\begin{itemize}
\item Código fuente en la rama correspondiente. 
\item Imágenes de prueba utilizadas (si es el caso)
\item Reporte escrito.
\end{itemize}
\end{document}
