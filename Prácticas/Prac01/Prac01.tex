\documentclass[letterpaper,11pt]{article}
\usepackage[spanish]{babel}
\spanishdecimal{.}
\usepackage[utf8]{inputenc}
\usepackage{graphicx}
\usepackage[top=2.5cm, bottom=3cm, left=2.5cm, right=3cm]{geometry}
\usepackage{hyperref}
\usepackage{verbatim}
\usepackage[natbibapa]{apacite}
\usepackage{multicol}
\usepackage{amsmath}
\usepackage{listings}
\usepackage{xcolor}
\bibliographystyle{apacite}

\definecolor{graywhite}{rgb}{0.9529,0.9607,0.9686}
\definecolor{bluegray}{rgb}{0.6823, 0.7411, 0.8}
\definecolor{darkred}{rgb}{0.7372, 0.2392, 0.2392}
\definecolor{bluedark}{rgb}{0.294, 0.4705, 0.6407}
\definecolor{darkgreen}{rgb}{0.1764, 0.5294, 0.0901}
\lstset{
  backgroundcolor=\color{graywhite},   % choose the background color; you must add \usepackage{color} or \usepackage{xcolor}; should come as last argument
  basicstyle=\footnotesize,        % the size of the fonts that are used for the code
  breakatwhitespace=false,         % sets if automatic breaks should only happen at whitespace
  breaklines=true,                 % sets automatic line breaking
  captionpos=b,                    % sets the caption-position to bottom
  commentstyle=\color{darkgreen},    % comment style
  keepspaces=true,                 % keeps spaces in text, useful for keeping indentation of code (possibly needs columns=flexible)
  keywordstyle=\color{darkred},       % keyword style
  language=Octave,                 % the language of the code
  morekeywords={*,...},            % if you want to add more keywords to the set
  numbers=left,                    % where to put the line-numbers; possible values are (none, left, right)
  numbersep=7pt,                   % how far the line-numbers are from the code
  numberstyle=\tiny\color{bluedark}, % the style that is used for the line-numbers
  showspaces=false,                % show spaces everywhere adding particular underscores; it overrides 'showstringspaces'
  showstringspaces=false,          % underline spaces within strings only
  showtabs=false,                  % show tabs within strings adding particular underscores
  stepnumber=1,                    % the step between two line-numbers. If it's 1, each line will be numbered
  stringstyle=\color{bluedark},     % string literal style
  frame=single,
  rulecolor=\color{bluegray},
  tabsize=2,                   % sets default tabsize to 2 spaces
  xleftmargin=1cm,
  xrightmargin=0.5cm,
  framexleftmargin=0.5cm,
  extendedchars=true,
  literate={á}{{\'a}}1 {é}{{\'e}}1 {í}{{\'i}}1 {ó}{{\'o}}1 {ú}{{\'u}}1 {Á}{{\'A}}1 {É}{{\'E}}1 {Í}{{\'I}}1 {Ó}{{\'O}}1 {Ú}{{\'U}}1,
}


\title{Práctica 01\\La biblioteca OpenCV}
\author{Visión Computacional Aplicada a la Robótica}
\date{Maestría en Ingeniería, UNAM, 2023-2}
\begin{document}
\renewcommand{\tablename}{Tabla}

\maketitle

\subsection*{Resumen}
El alumno realizará un programa para agregar un background virtual al video obtenido de una cámara web. 

\subsection*{Objetivos}
\begin{itemize}
\item Aprender a abrir imágenes y flujos de video mediante la biblioteca OpenCV
\item Aprender a manipular las estructuras de datos utilizadas por Numpy y OpenCV
\item Familiarizarse con el uso de las herramientas de interfaz gráfica de OpenCV
\end{itemize}

\section*{Duración}
1 semana

\section*{Desarrollo}
Escriba un programa en Python que realice lo siguiente:
\begin{enumerate}
\item Abrir una WebCam y mostrar el video en una ventana.
\item Capturar con el cursor del mouse un \textit{bounding box}, de modo que el usuario pueda seleccionar una región de interés en la imagen utilizando el mouse.
\item Determinar el promedio de color $\bar{x}$ de los pixeles dentro del \textit{bounding box}
\item Agregar un \textit{track bar} a la ventana de video con la que el usuario pueda seleccionar una tolerancia $\Delta x$
\item Determinar el conjunto de pixeles $P$ cuyo color esté en el intervalo $\bar{x} \pm \Delta x$
\item Sustituir todos los pixeles $P$ del video por los correspondientes de una imagen \textit{background} (el alumno puede usar la imagen que desee)
\item Mostrar el video resultante en otra ventana.
\end{enumerate}



\subsubsection*{Funciones de referencia}
Se pueden utilizar las siguientes funciones de OpenCV (son funciones sugeridas, utilice las que considere convenientes):
\begin{multicols}{3}
  \begin{itemize}
  \item cv2.imread
  \item cv2.bitwise\_and
  \item cv2.bitwise\_or 
  \item cv2.bitwise\_not
  \item cv2.bitwise\_xor
  \item cv2.inRange
  \item cv2.merge
  \item cv2.split
  \item cv2.imshow
  \item cv2.VideoCapture
  \item cv2.mean
  \end{itemize}
\end{multicols}

\section*{Entregables:}
\begin{itemize}
\item Subir el código fuente al repositorio en la rama correspondiente.
\item Sobreescriba el archivo que se encuentra en la ruta \texttt{catkin\_ws/src/students/scripts/practice01.py}
\end{itemize}
\end{document}
